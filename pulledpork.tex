\section{South Carolina Pulled Pork}
\begin{recipe}

\ingredients{
                5 & pound pork shoulder \\
                3 & tablespoons mustard powder \\
                2 & tablespoons salt \\
    1\sfrac{1}{2} & tablespoons brown sugar \\
                2 & teaspoons pepper \\
                2 & teaspoons paprika \\
     \sfrac{1}{4} & teaspoon cayenne \\
}

Pat meat dry with paper towels. Rub spices into the meat, cover in plastic wrap and store in the fridge for two days.

\ingredients{
    8 & cups wood chips \\
}

Soak the wood chips in water for 15 minutes.

Make 3 foil packets of wood chips, poke holes in the side of the foil.

\ingredients{
    50 & charcoal briquettes \\
    15 & more charcoal briquettes \\
}

Light the briquettes in a chimney starter. After 10 minutes, before the coals are covered in gray dust, pour them out over just half of the grill.
It's OK if some of them are un-lit. Add the 15 remaining un-lit coals on top for a slow steady burn.

Place a foil packet on top of the coals. Place the pork on the cold side of the grill.
Make sure the bottom vents are open, and the top vents are half open.

Roast for two hours, adding a wood chip packet every 30 minutes, and then remove to a roasting pan or disposable foil tray.

\ingredients{
    \sfrac{1}{2} & cup yellow mustard \\
    \sfrac{1}{2} & cup brown sugar \\
    \sfrac{1}{4} & cup white vinegar \\
               2 & tablespoons Worcestershire sauce \\
               1 & tablespoon hot sauce \\
               1 & tablespoon salt \\
               1 & tablespoon pepper \\
}

Whisk the ingredients, and brush half onto the pork.

Cover tightly with tin foil and roast in a 325\degree{} oven for 3 hours.

The pork is done when you can plunge a fork into it with no resistance.

You can either shred the pork right away, or let it sit until it's cool enough to handle. Don't worry about resting it; all of the juices have left the meat.

Shred with two forks, and toss with the remaining sauce and pork juices.

\end{recipe}
