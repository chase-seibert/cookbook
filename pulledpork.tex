\section{South Carolina Pulled Pork}
\begin{recipe}

\pre{
    Pork shoulder is also called Boston butt or pork butt.

    You can use the bone-in variety, but you will need to remove the fat cap and allow extra cooking time.
}

\ingredients{
                3 & tablespoons mustard powder \\
                2 & tablespoons salt \\
    1\sfrac{1}{2} & tablespoons brown sugar \\
                2 & teaspoons pepper \\
                2 & teaspoons paprika \\
     \sfrac{1}{4} & teaspoon cayenne \\
}

Mix spices.

\ingredients{
    5 & pound pork shoulder \\
}

Pat meat dry with paper towels. Rub spices into the meat, cover in plastic wrap and store in the fridge over night.

\columnbreak

\ingredients{
    4 & cups wood chips \\
}

Soak the wood chips in water for 15 minutes.

Wrap the wood chips tightly in tin foil, and poke holes in the side of the foil with a pairing knife.

\ingredients{
    50 & charcoal briquettes \\
}

Light the briquettes in a chimney starter. After about 20 minutes, when the coals are covered in gray dust, pour them out over just half of the grill.

Place the foil packet on top of the coals.

Place the pork on the cold side of the grill. Make sure the bottom vents are open, and the top vents are half open.

Roast for two hours, and then remove to a roasting pan or disposable foil tray.

\ingredients{
    \sfrac{1}{2} & cup yellow mustard \\
    \sfrac{1}{2} & cup brown sugar \\
    \sfrac{1}{4} & cup white vinegar \\
               2 & tablespoons Worcestershire sauce \\
               1 & tablespoon hot sauce \\
               1 & tablespoon salt \\
               1 & tablespoon pepper \\
}

Whisk the ingredients, and brush half onto the pork.

Cover tightly with tin foil and roast in a 325\degree oven for 3 hours.

The pork is done when you can plunge a fork into it with no resistance.

You can either shred the pork right away, or let it sit until it's cool enough to handle. Don't worry about resting it; all of the juices have left the meat.

Shred with two forks, and toss with the remaining sauce.

\end{recipe}
