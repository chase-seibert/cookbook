\section{Lasagna}

\begin{recipe}

\pre {
    Moisture is the enemy. That's why I use no-bake noodles. They soak up a lot, plus
    their texture is closer to fresh noodles.
}

\ingredients{
               4 & tablespoons butter \\
    \sfrac{1}{4} & cup flour \\
               4 & cups whole milk \\
               1 & teaspoon kosher salt \\
               8 & ounces Parmesan cheese  \\
}

To create the bechamel sauce, melt butter in a sauce pan. Add flour and whisk. Cook for two minutes, stirring frequently. Slowly whisk in milk. Add salt. Increase heat to medium high and simmer for 10 minutes, stirring occasionally. Melt in Parmesan. It might be grainy, but you don't taste it in the final dish.

\ingredients{
    15 & ounces ricotta \\
     1 & cup fresh basil \\
     1 & egg \\
       & fresh pepper \\
}

Combine all items in a bowl.

\ingredients{
     1 & box no-bake Lasagna noodles \\
    15 & ounces mozzarella \\
       & tomato or ragu sauce; see bellow recipe \\
}

Combine tomato sauce and bechamel. Pour a large ladle of sauce in the bottom of a 9x13 baking dish, and spread it out.

Lay down a layer of noodles, followed by a layer of the ricotta mixture, a layer of sauce and a layer of grated mozzarella.

Repeat with a second layer.

\ingredients{
    10 & ounces Gruyere cheese \\
     1 & tablespoons garlic powder \\
}

Top with a final layer of noodles, and the grated Gruyere.

Bake at 375\degree{} for 40 minutes.

Remove from the oven, and let cool for 30 minutes before serving.

\newpage

\subsection{Meat Sauce}

\pre{
    Meatloaf mix is sold as such in most grocery stores; it's a mix of beef, veal and pork. You can also substitute 80 percent beef.
}

\ingredients{
    28 & ounces diced tomatoes \\
    56 & ounces tomato puree (2 cans) \\
     1 &  pound meatloaf mix \\
     1 &  cup whole milk \\
     5 &  cloves garlic \\
     1 &  tablespoon tomato paste \\
     1 &  tablespoon red pepper flakes \\
     1 &  large onion \\
}

Chop onion fine, and sauté in oil until just starting to brown. Add garlic and tomato paste and cook for 3 minutes.
Add meat and stir until it's lost the red color. Add milk and bring to simmer. Add canned tomatoes.
Simmer uncovered for 3 hours.

\subsection{Homemade Ricotta}

\ingredients{
               1 & gallon milk \\
    \sfrac{3}{4} & cups lemons juice(6 lemons) \\
               2 & teaspoons kosher salt \\
}

Bring milk and salt to 185\degree{} in a large stock pot. Remove from heat, and stir in lemon juice. Allow to rest for 20 minutes.

Pour mixture through a cloth stretched over a colander, and let drain for an hour. Cover with plastic wrap, and refrigerate for between 10 and 24 hours.

If the milk does not separate into curds and whey within 5 minutes of adding the lemon juice, add a couple more tablespoons and wait another 5 minutes. You can substitute plain white vinegar for some of the lemon juice, if you run short.

\end{recipe}
