
\section{Lasagna}
\begin{recipe}

\pre{
    I don't think homemade noodles are superior. I prefer De Cecco   dried noodles (regular, not no-bake).
}

Preheat oven to 375 degrees.

\ingredients{
    1 & box Lasagna noodles \\
      & (about 20 noodles) \\
}

Add noodles one at a time into boiling water, to minimize sticking. Remove when they are al dente.

\ingredients{
    15 & ounces ricotta \\
     5 & ounces grated Parmesan \\
     1 & cup fresh basil \\
     1 & egg \\
       & fresh pepper \\
}

Combine all items in a bowl.

\ingredients{
    15 & ounces mozzarella \\
}

Pour a large ladle of sauce in the bottom of a 9x13 baking dish, and spread it out.

Lay down a layer of noodles, followed by a layer of the ricotta mixture, a layer of sauce and a layer of grated mozzarella.

Repeat with a second layer.

Top with another layer of noodles, and the remaining mozzarella. Cover with tin foil, and bake at 375 degrees for 15 minutes. Remove foil, and bake another 25 minutes.

Remove from the oven, and let cool for 10 minutes before serving.

\subsection{Meat Sauce}

\tip{
    Meatloaf mix is sold as such in most grocery stores; it's a mix of beef, veal and pork. You can also substitute 85 percent lean beef.
}

\ingredients{
    28 & ounces diced tomatoes \\
    28 & ounces tomato puree \\
     1 &  pound meatloaf mix \\
     5 &  cloves garlic \\
     1 &  tablespoon tomato paste \\
     1 &  tablespoon red pepper flakes \\
     1 &  large onion \\
}

Chop onion fine, and sauté in oil until just starting to brown. Stir in tomato paste, red pepper flakes and minced garlic, for 30 seconds. Stir in meat, breaking into small pieces.

When the meat just starts to loose its color, stir in tomatoes and simmer uncovered for at least 20 minutes, or up to 3 hours. The longer, the better.

\clearpage

\subsection{Homemade Ricotta}

\tip{
    This is not really ricotta; it's slightly sweeter, much drier and can be used interchangeably with ricotta.

    I prefer real cloth (jersey) to cheesecloth, as the ricotta tends to get stuck inside  cheesecloth.
}

\ingredients{
               1 & gallon milk \\
    \sfrac{3}{4} & cups lemons juice(6 lemons) \\
               2 & teaspoons kosher salt \\
}

Bring milk and salt to 185 degrees in a large stock pot. Remove from heat, and stir in lemon juice. Allow to rest for 20 minutes.

Pour mixture through a cloth stretched over a colander, and let drain for an hour. Cover with plastic wrap, and refrigerate for between 10 and 24 hours.

If the milk does not separate into curds and whey within 5 minutes of adding the lemon juice, add a couple more tablespoons and wait another 5 minutes. You can substitute plain white vinegar for some of the lemon juice, if you run short.

\subsection{Garlic Bread}

\tip{
    This recipes coats the bread after it comes out of the oven. Most garlic bread recipes coat the bread before it goes into the oven, which leads to soggy bread.
}

\ingredients{
    8 & garlic cloves \\
    1 & tablespoon veg oil \\
}

Mince garlic, combine with oil in a small bowl a microwave on low for up to a minute. The garlic should mellow and loose some of its raw taste.

\ingredients{
               5 & tablespoons butter (room temperature) \\
    \sfrac{1}{4} & cup fresh parsley \\
}

Combine garlic, butter and parsley in a small bowl.

\ingredients {
    1 & loaf French bread \\
}

Cut the bottom off the loaf, and then halve the loaf lengthwise. Place on a baking sheet in a 450 degree oven for 10 minutes, or until well browned.

\ingredients {
    \sfrac{1}{2} & cup finely grated Parmesan \\
}

Remove bread from the oven, slather with garlic butter mixture, top with Parmesan and wrap tightly in aluminum foil.
After 5 minutes, bread can be sliced while still inside foil and served.

\end{recipe}
