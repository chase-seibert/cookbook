\section{Thanksgiving Dinner}
\begin{recipe}

The turkey temp should be 165\degree{} for the white meat, and 175\degree{} for the dark meat. It's OK if both go 10-15\degree{} over that. Recipes are good for up to 12 adults and 5 young children.

\ingredientsLeft {
    & 9:00 start green bean base \\
    & 10:00 stuffing \\
    & 10:30 start gravy \\    
    & 11:00 ice turkey \\
    & 11:15 peel potatoes into cold water \\
    & 12:00 turkey in at 425\degree{} \\
    & 12:00 put out ice for drinks \\
    & 1:00 add stuffing, 325\degree{} \\
    & 1:00 rotate turkey \\
    & 1:30 peel potatoes into cold water \\    
    & 2:30 put potatoes on heat \\
    & 2:30 start dishwasher so you can empty after dinner \\
    & 3:00 turkey out \\
    & 3:00 crescent rolls in at 425\degree{} (no traffic jam at the end) \\    
    & 3:15 finish gravy \\
    & 3:30 assemble green beans and put in oven 425\degree{}  \\
    & 4:00 Dinner \\
}

\newpage
\subsection{Green Bean Casserole}

\ingredients{
    6 & slices white bread \\
    2 & tablespoons butter \\
    1 & can fried onions \\
}

Pulse bread and butter in food processor with salt and pepper. Toss with fried onions.

\ingredients {
    2              & pounds green beans \\
}

Trim and cut green beans into thirds. Smaller pieces are better. 
Boil for 6 minutes and cool in an ice water bath.

\ingredients {
    2              & pound button mushrooms \\
    1              & pound Shiitake mushrooms \\
    5              & tablespoons butter \\
    3              & garlic cloves \\
    3              & tablespoons flour \\
    4 & cups chicken broth \\
    2 & cups heavy cream \\
    1 & teaspoon cider vinegar \\ 
}

Sauté mushrooms, broken in to pieces, with butter and salt. 
Cover the pot while the mushrooms sweat down. 
Add garlic. Add flour and cook for one minute. 
Add liquids and simmer for 3 hours to reduce. 
Finish with vinegar and season to taste. 
It's OK if it's liquidy. Toss with green beans. 
Place into 9x13 baking dish. 
Don't add topping until it's ready to go into the oven.

Bake at 425\degree{} for 15 minutes. 30 minutes if it starts cold. 

\newpage
\subsection{Stuffing}

\tip {
    Try using two different kinds of bread. I've had good luck with Brioche, challah, sourdough and rye. Don't stretch it to three loaves, the stuffing will be dry. Pre-sliced bread is OK. 
}

\ingredients{
    2 & loaves of bread \\
}

Cube bread into \sfrac{1}{4} inch pieces. Toast in the oven at 325\degree{} for 30 minutes, until it's very dry and toasted. Instead of putting multiple pans in the oven at once or crowding the pans, take your time and do four separate batches. Different types of bread will brown at different rates. 

\ingredients{
    2 & onions \\
    4 & ribs of celery \\
    3 & cups chicken broth or turkey stock \\
    3 & tablespoons butter \\
      & salt \\
      & pepper \\
    1 & teaspoon cider vinegar \\ 
}

In dutch oven, sauté onion and celery in butter with salt until just starting to brown. Add broth and bring to simmer. If you want to add seasonings, rosemary, thyme, and fennel are all good. 
Finish with vinegar and season with salt aggressively. 
Toss with bread cubes until the broth is absorbed. It's OK if it's a little dry; it will absorb juices from the turkey. 

\newpage
\subsection{Gravy}

\tip {
    I think this recipe for gravy without using the pan drippings is even better than with the pan drippings, and it allows you to absorb all the drippings into the stuffing. 
}

\ingredients{
    1 & onion \\
    4 & celery ribs \\
    8 & tablespoons butter (one stick) \\
      & reserved turkey pieces \\
}

In a dutch oven, melt butter and add onion and celery. Brown on medium high heat until very dark, but not burnt. Add more butter as necessary so that their is plenty of fat for browning. This step can take up to an hour. Add chopped up turkey gizzard, kidney and neck. Skip just the liver. 

\ingredients {
    12 & cups turkey or chicken broth \\
      & reserved turkey liquid \\
    1 & teaspoon cider vinegar \\ 
}

Add some broth, and cover. Bring to a simmer, to start rendering the fat in the turkey neck. 
Let it reduce down until dark brown. 

Change heat to low and continue to stir occasionally for the rest of the day (3-4 hours). 
Add the fat and liquid from the reserved pan drippings when available. 
Continue to add broth as needed to keep from burning, but don't add all the liquid at once. 
Gravy should be a dark brown.

Finish gravy before dinner by straining over a bowl to remove any solids, and return the gravy to the dutch oven. Don't strain out the fat. You can continue to add stock here to produce a larger amount of gravy. Just bring it to a simmer again. 

\ingredients {
    1 & cup water \\
    1 & tablespoon corn starch \\
}

Mix cornstarch and water in a glass until cornstarch is dissolved. Slowly whisk into gravy and bring to simmer, until thickened and glossy. 

\subsection{Mashed Potatoes}

\tip {
    5 pounds of mashed potatoes is fine for 18 people. 10 pounds is a ridiculous amount. Low fat buttermilk doesn't work, it will curdle. Substitute sour cream instead. 
}

\ingredients{
    5 & pounds russet potatoes \\
    1 & stick butter \\
    1 & cup buttermilk \\
      & chives or scallions \\
}

Start in cold water and boil for 20 minutes, longer is fine. User a ricer to break them down. Add a butter in chunks and stir in buttermilk gradually. Don't pre-heat buttermilk or it will curdle. Mashed potatoes can be kept warm by placing in the microwave. 

\newpage
\subsection{Turkey}

\pre{
    A 12 pound turkey is plenty for eight people. 22 pounds is plenty for 18 people. For up to 18 pounds, you don't need to increase the cooking time at all. Icing and flipping the tukey is worth it for well cooked dark meat. Let the turkey rest in the oven to give the dark meat even more time to come up to temperature. 
}

\ingredients{
                     & 12 pound turkey \\
    \sfrac{1}{2} & cup kosher salt \\
}

48 hours before roasting, unwrap and separate gizzard and neck pieces. Salt liberally and use a knife to create fat draining slits on the dark meat area. 

One hour before roasting remove from refrigerator and put ice on the breasts. 

Place in roasting pan on a raised rack, upside down, on some foil so it doesn't stick. Brush with melted butter, and roast at 425\degree{} for one hour.

Lower oven to 325\degree{}. Remove turkey in the raised rack, and flip is over, removing foil. Empty any liquid in the pan into a bowl and reserve for gravy.

Place the stuffing into the bottom of the roasting pan. Press the stuffing down to compress. Put the turkey on top of the stuffing, and return to oven for another two hours. If it's cooking too fast, you can turn off the oven and prop open the door, and it will keep climbing 10\degree{}. You should remove the turkey from the oven when the breast reads 170\degree{}. This means it will climb to 175\degree{} or 180\degree{}; but if it's well salted you don't need to worry about overcooking it. A remote thermometer is critical. 

\subsection{Carving}

Drumsticks and wings should be removed first, by slipping a blade in between the knob joints.

The thighs are where most of the good dark meat comes from. It's between the drumsticks and the main body. There is a second joint there which must be separated, and then the dark meat can be removed from the bone. On a well cooked bird, you can remove the bone easily with just your fingers.

Remove the breasts before carving with sweeping motions between the backbone and the breast meat.
Once the breasts are separated, they can be sliced vertically on a cutting board.
I prefer them in one inch slices.

\end{recipe}
