\section{Turkey, Stuffing and Gravy}
\begin{recipe}

\pre{
    I don't worry about the bird drying out in the oven. Salting will prevent this for the most part, and I find that trying to cook it until just barely the right temperature tends to leave the meat soggy and squishy. I prefer the fat to be well rendered, which requires going well past 165\degree. I have had white meat register up to 195\degree and still be moist.
}

\subsection{Stuffing}

\tip {
    Try using two different kinds of bread. I've had good luck with Brioche and sourdough.
}

\ingredients{
    2 & loafs bread \\
}

Cube bread into one inch pieces. Toast in the oven at 250\degree for 20 minutes, until it's very dry.

\ingredients{
    1 & onion \\
    4 & ribs of celery \\
    2 & cups chicken broth \\
      & butter \\
      & salt \\
      & pepper \\
}


In dutch oven, sauté onion and celery in butter with salt until just starting to brown. Add broth and bring to simmer. Toss with bread cubes until the broth is absorbed.

\subsection{Turkey}

\tip{
    A 12 lb turkey is plenty for eight people.
}

\ingredients{
                     & 12 lb turkey \\
    \sfrac{1}{3} & cup kosher salt \\
}

At least 48 hours before roasting, remove turkey from wrapping and place in the sink. Wash with water, and remove and reserve gizzard and neck pieces.

Create half inch slits with a knife at the ends of the breasts and the drumsticks. Using the handle of a wooden spoon, separate the skin from the meat underneath.

Using your hands, rub a generous amount of salt directly onto the meat. Err on the side of over-salting, versus under salting. Also salt inside the main cavity.

Tent loosely with a paper towel and put in the refrigerator.

Place in roasting pan on a raised rack, and roast at 425\degree for one hour.

Lower oven to 325\degree. Remove turkey, and set the inner rack on a cutting board. Empty any liquid in the pan into a bowl and reserve for gravy.

Place the stuffing into the bottom of the roasting pan and return to oven for another two hours.

\subsection{Gravy}

\tip {
    I think this recipe for gravy without using the pan drippings is even better than with the pan drippings, and it allows you to absorb all the drippings into the stuffing.
}

\ingredients{
    1 & onion \\
    4 & celery ribs \\
      & reserved turkey pieces \\
}

In a dutch oven, melt butter and add onion and celery. Brown on medium high heat until very dark, but not burnt.

Add chopped up turkey gizzard, kidney and neck. Sauté until dark brown.

\ingredients {
    2 & cups chicken broth \\
      & reserved turkey liquid \\
}

De glaze with chicken broth, and add any reserved roasting liquid. Gravy should be a dark brown. If you have any remaining herbs, add them now.

Bring to a fast simmer and reduce. Strain over a bowl to remove any solids, and return the gravy to the dutch oven.

\ingredients {
    1 & cup water \\
    1 & tablespoon corn starch \\
}

Mix cornstarch and water in a glass until cornstarch is dissolved. Slowly which into gravy and bring to simmer to thicken.

Season to taste with salt and pepper.

\subsection{Mashed Potatoes}

\ingredients{
    5 & lbs russet potatoes \\
    1 & stick butter \\
    1 & cup buttermilk \\
}

Steam one inch cubes for 15 minutes until easily pierced by a pairing knife. User a ricer to break them down. Add a butter in chunks and stir in buttermilk.

\subsection{Carving}

Drumsticks and wings should be removed first, by slipping a blade in between the knob joints.

The thighs are where most of the good dark meat comes from. It's between the drumsticks and the main body. There is a second joint there which must be separated, and then the dark meat can be removed from the bone. On a well cooked bird, you can remove the bone easily with just your fingers.

Remove the breasts before carving with sweeping motions between the backbone and the breast meat. Once the breasts are separated, they can be sliced vertically on a cutting board. I prefer them in one inch slices. Tough it's not how most people expect it to be served, larger pieces leave much more moisture in the meat.

\end{recipe}
