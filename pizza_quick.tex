\section{Quick Pizza}
\begin{recipe}

\pre{
    Many supermarkets sell pre-made pizza dough. This dough can be frozen, even
    if it is fresh in the supermarket. Defrost by moving to the refrigerator
    two days in advance.

    Pre-heating the pizza stone for 45 minutes will improved the crust.
}

\ingredients{
    1 & pizza dough \\
}

Turn into a bowl well-coated with olive oil, cover in plastic wrap, and let
come to room temperature for two hours. You can also place in a 120\degree{} oven
for 1 hour.

Cut a large square of parchment paper, and empty pizza dough onto it. Lift the
edges with your hands and stretch it out a little. Then use the palms of your
hands to spread it further.

\ingredients{
      1 & teaspoon oregano \\
      1 & small can tomato paste \\
      15 & ounces diced tomato \\
}

Sauté tomato paste in olive oil over low heat for 10 minutes. Add
diced tomatoes, blend with a stick blender, and simmer for 10 minutes.

\ingredients{
    15 & ounces shredded mozzarella \\
     6 & ounces pepperoni \\
}

Cover dough with sauce, cheese and pepperoni. Cook on a pizza stone at 550\degree{} for 8 minutes
(6 minutes if you use convection, which results in a better char)
Let it cook until it looks like it's just turning dark brown. Remove from oven
and cool on wired rack.

\tip{
    You can re-heat pizza by folding inside a aluminum foil envelope and placing
    in a 275\degree{} over for 25 minutes.
}

\subsection{Pizza Dough}

\ingredients{
      1 \sfrac{3}{4} & cup warm water \\
      1 & packet instant yeast \\
      2 & tablespoons olive oil \\
     4 & cups bread flour \\
      2 & tablespoons kosher salt \\
}

Pulse flour and salt in a food processor. Mix wet ingredients together and
let stand for 5 minutes. Slowly add the wet ingredients to the flour with the
food processor running. Run it for another 60 seconds until it comes into a
ball. Remove, kneed briefly and divide in two. Dough can either be set out
to rise or frozen for later.

\end{recipe}
