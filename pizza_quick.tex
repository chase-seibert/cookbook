\section{Quick Pizza}
\begin{recipe}

\pre{
    Many supermarkets sell pre-made pizza dough. This dough can be frozen, even
    if it is fresh in the supermarket. Defrost by moving to the refrigerator
    ONE day in advance.
}

\ingredients{
    1 & pizza dough \\
}

Turn into a bowl well-coated with olive oil, cover in plastic wrap, and let
come to room temperature for two hours. If you don't have time for that, you
can put it in a plastic bag w/ olive oil, remove as much air as possible, and
submerge in warm water for 30 minutes.

Cut a large square of parchment paper, and empty pizza dough onto it. Lift the
edges with your hands and stretch it out a little. Then use the palms of your
hands to spread it further. Don't spread it too thin; you want some chew in
the final crust.

\ingredients{
      15 & ounces crush tomato (ideally San Marzano) \\
      1 & teaspoon oregano \\
      1 & teaspoon sugar \\
      \sfrac{1}{2} & anchovy \\
      2 & garlic cloves \\
}

Just blend it with an immersion blender, right in the can. It will make enough
for 3-4 small pizzas. The anchovy is a critical ingredient, but it's also easy
to make it too fishy!

\ingredients{
     6 & ounces shredded mozzarella \\
     2 & ounces shredded Parmesan \\
     6 & ounces pepperoni \\
}

Cover dough with sauce, cheese and pepperoni. There is no such thing as too much
pepperoni.

Assuming you pre-heated with the broiler (then turned it off), you should be able
to cook the pizza in just 5 minutes. Without the broiler pre-heat, it will take
8 minutes. The faster you can cook it, the better the pizza will be.

Remove from oven and cool on wired rack.

\subsection{Pizza Dough}

\ingredients{
      1 \sfrac{3}{4} & cup warm water \\
      1 & packet instant yeast \\
      2 & tablespoons olive oil \\
     4 & cups bread flour \\
      2 & tablespoons kosher salt \\
}

Pulse flour and salt in a food processor. Mix wet ingredients together and
let stand for 5 minutes. Slowly add the wet ingredients to the flour with the
food processor running. Run it for another 60 seconds until it comes into a
ball. Remove, kneed briefly and divide in two. Dough can either be set out
to rise or frozen for later.

\end{recipe}
