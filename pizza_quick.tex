\section{Quick Pizza}
\begin{recipe}

\ingredients{
    1 & pizza dough \\
}

Turn into a bowl well-coated with olive oil, cover in plastic wrap, and let
come to room temperature for two hours. If you don't have time for that, you
can put it in a plastic bag w/ olive oil, remove as much air as possible, and
submerge in warm water for 30 minutes.

Cut a large square of parchment paper, and empty pizza dough onto it. Lift the
edges with your hands and stretch it out a little. Then use the palms of your
hands to spread it further. The thinner you can get it, the better. Almost
translucent in the middle is best!

\ingredients{
        & olive oil \\
     15 & ounces tomato (ideally San Marzano) \\
      2 & garlic cloves \\
        & red pepper flakes \\
}

Saute garlic and pepper flakes in olive oil briefly. Add tomatoes. Simmer for 10 minutes on high.
If there are large chunks, blend with an immersion blender.

\ingredients{
     6 & ounces shredded mozzarella \\
     2 & ounces shredded Parmesan \\
     6 & ounces pepperoni \\
}

Cover dough with sauce, cheese and pepperoni. There is no such thing as too much
pepperoni! The cheese may look sparse or have gaps, but it will fill in when it
bakes. Less cheese is actually better for browning.

You should be able to cook the pizza in just 10 minutes.
The faster you can cook it, the better the pizza will be.
You want the crust to be blackened in spots, and the cheese
to be mostly browned.

Remove from oven and cool on wired rack.

\subsection{Pizza Dough}

\ingredients{
      2 & cup warm water \\
      1 & packet instant yeast \\
      4 & tablespoons olive oil \\
      3 & cups bread flour \\
      2 & tablespoons kosher salt \\
}

Manually mix everything except the water in a large, air tight container. Add water until it's too wet, then add in a little more flour - mixing well. When it's loose and sticky but not a puddle put the lid on the container and let it rest at room temperature for 3-4 hours. Then put in the fridge.

Dough is best if you let it ferment in the fridge for 3-4 days.

When you're ready to use, do some light kneeding without folding the dough over on itself. This will develop more texture.

\end{recipe}
