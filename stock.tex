\section{Stock - Chicken, Vegetable}
\begin{recipe}

\pre{
    Stock is very simple to make. It's only slightly harder to make well!
}

\ingredients{
    2 & onions \\
    4 & celery stalks \\
    2 & carrots \\
}

Roughly chop these ingredients. They don't have to be peeled at all, even the onions. Cover with four quarts of water.

\tip{
    You can also add any protein and/or bones you have on hand. A chicken or turkey carcase is the usual option.
}

Ingredients should be covered by the water. This may require the larger items to be broken down first.

Bring to boil on the stove, and quickly reduce to the lowest heat setting. You want the surface to be broken by a small bubble only once every two or three seconds.

Simmer for at least four hours.

\ingredients {
    1 & packet powdered gelatin \\
    1 & bunch herbs \\
      & kosher salt \\
      & pepper \\
}

Time to flavor the stock. Add liberal amounts of salt and pepper to taste.

\tip {
    The gelatin serves to give the stock a better mouth feel. You can get the same effect by using a lot of bones; gelatin with naturally be extracted from the bone marrow.
}

Remove one cup of the stock and mix with gelatin powder, then add the gelatin mixture back to the stock.

Typical herbs to use would be rosemary, thyme and sage. The them into a bundle and steep in the stock.

Filter the stock trough a fine mesh strainer like a Chinois. and discard solids. Stock can be used immediately, or cooled at room temperature for a couple of hours and then refrigerated. Should be used within two weeks.

\end{recipe}

%\fullpageimage{stock.jpg}{http://www.flickr.com/photos/iguana\_azul/}
