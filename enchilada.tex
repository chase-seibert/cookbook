
\fullpageimage{enchilada.jpg}{http://www.flickr.com/photos/foodista}

\section{Red Chile Chicken Enchiladas}
\begin{recipe}

\ingredients{
    1 & tablespoon veg oil \\
    1 & onion \\
}

Chop onion fine, and sauté in a Dutch oven over medium heat until just starting to brown.

\ingredients{
    3 & garlic cloves \\
    3 & tablespoons chili \\
    2 & teaspoons coriander \\
    2 & teaspoons cumin \\
    2 & teaspoons sugar \\
    1 & teaspoon oregano \\
    1 & teaspoon kosher salt \\
}

Mix ingredients, sauté for 30 seconds on high, then add to onions. Stir for 30 seconds.

\ingredients{
    2 & chicken breasts \\
}

Remove excess fat, and cut into ¼ inch strips. Add to onions and spices, and stir until coated.

\ingredients{
    1\sfrac{1}{4} & cup tomato sauce \\
    1\sfrac{1}{4} & ounces heavy cream \\
    3\sfrac{1}{2} & ounces chipotle (in adobe sauce) \\
                1 & tablespoon lime \\
                  & juice \\
}

Add liquids, and simmer for 10 minutes. Pour mixture through a strainer, extracting as much sauce as possible. Reserve sauce and solids.

Chill the chicken for 10 minutes in the freezer.

Season red sauce to taste with salt, pepper and lime juice. Simmer for 10 more minutes.

\ingredients{
    \sfrac{3}{4} & pound jack cheese \\
    \sfrac{1}{2} & pound mild cheddar \\
               1 & cup fresh cilantro \\
              12 & ounces pickled \\
                 & jalapeños \\
}

Chop jalapeños and cilantro in a small Cuisinart. Grate cheese, and combine all ingredients in a mixing bowl, along with the solids from the sauce. Combine with chicken.

\ingredients{
    10 & ten inch flour \\
       & tortillas \\
}

Microwave tortillas for a minute to soften. Coat the bottom of a 9x13 baking dish with the red sauce.

One at a time, dip a tortilla in the sauce, fill with chicken and cheese mixture, roll up and place in the baking dish. Cover the dish with red sauce. There should be no dry tortilla.

Top with more cheese and bake uncovered at 375 degrees for 25 minutes. Rest for 10 minutes before serving.

Serve with sour cream and lime wedges. Pairs well with Spanish rice and refried beans.

\end{recipe}
