
\fullpageimage{chili.jpg}{http://www.flickr.com/photos/40389150@N06}

\section{Noel's Chili}
\begin{recipe}

\pre{
    The flavor compounds in chili powder are fat soluble; you need to saute them
    in fat. You want beef with a high fat content, not 90+ percent lean.
}

\ingredients{
      1 & tablespoon vegetable oil \\
    1.5 & pound 80 percent lean beef \\
}

Brown over medium high heat in a cast iron skillet.

\ingredients{
    1 & tablespoon vegetable oil \\
    1 & large onion \\
    1 & red bell pepper \\
    1 & green bell pepper \\
    3 & garlic clove \\
    3 & tablespoons ancho chili powder \\
    3 & tablespoons taco seasoning \\
    1 & teaspoon oregano \\
}

In a large dutch oven, sauté finely diced onions until just browning. Add diced peppers and smashed garlic cloves. Drain the rendered fat from the beef into the onions and peppers, and add chili and taco seasoning.


\ingredients{
    1 & beer \\
    1 & cup chicken broth \\
    1 & can chili beans in sauce \\
    1 & can diced tomatoes \\
}

Add beer to the onion mixture to deglaze. Scrape up all the bits. Add the other ingredients, along with the beef and any fat that has been rendered. Do not drain the fat!

Place the chili, uncovered, in a 325 degree oven for 90 minutes.

\ingredients{
    1 & cups white rice \\
}

Start the rice, and increase the oven temperature to 375 if there is still too much liquid in the chili pot. You want it to look like a well concentrated gravy, not a soup.

\ingredients{
    & lime \\
    & parsley \\
}

After removing chili from the oven, stir in the juice of one lime and minced parsley. If you have a green adobo ready, you can use that instead of plain parsley.

\ingredients{
    & sharp cheddar \\
    & sour cream \\
    & more onion \\
    & tortilla strips \\
}

Serve over rice with grated cheese, sour cream, raw onion and tortilla strips.

\end{recipe}
