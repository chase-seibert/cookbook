
\section{Perfect Omelet}
\begin{recipe}

\pre{
    Eggs want to be handled and heated as little as possible for maximum tenderness. That is why I start with room temperature eggs.

    Developing a curd by swirling the eggs for a few seconds after they have hit the pan adds texture.

    Many cooks keep a separate omelet pan. It's not strictly necessary, but I prefer a cast-iron pan with very low sides.

    Finally, don't add too much filling. Keep the filling to a minimum, and pre-cook any filling except cheese.
}

\columnbreak

\ingredients{
    1 & bell pepper \\
    1 & small onion \\
}

Dice fine, and pre-cook over medium heat until the onions start to brown. Remove from heat and reserve.

\ingredients{
    3 & eggs, room temperature \\
    1 & teaspoon dried dill \\
      & pinch kosher salt \\
}

Crack eggs into a medium bowl, salt, and whisk lightly with dill.

\ingredients{
    1 & tablespoon butter \\
}

Heat the pan over medium heat for five minutes. Coat with butter, removing excess, and wait a few minutes until the butter stops bubbling.

Pour egg mixture into the pan, and immediately stir with a spatula. In just a few seconds, a few large curds will have formed. Shake the pan to redistribute raw egg into any gaps.

\ingredients{
    1 & ounce feta cheese \\
}

Top two thirds of the omelet with cheese, pepper and onion. Cook until the raw egg has firmed up everywhere, and bubbles have started to form.

Run the tip of a spatula around the rim of the omelet, separating it from the pan.

Give the pan a firm shake, dislodging the omelet from the pan surface.

Using the spatula, flip the third of the omelet with no filling onto itself.

Shaking the pan, slide the folder end of the omelet onto a plate, and flip the remaining part over the top.

\end{recipe}

\fullpageimage{omelet.jpg}{http://www.flickr.com/photos/stone-soup}

