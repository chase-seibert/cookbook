
\fullpageimage{potroast.png}{http://www.flickr.com/photos/fattytuna}

\section{Pot Roast}
\begin{recipe}

\pre{
    Variations on this recipe start the same, and then vary the cooking liquid.
}

\ingredients{
    4 & lb chuck eye roast \\
}

Sprinkle liberally with kosher salt, and rest overnight in the refrigerator, lightly tented with paper towel.

Before cooking, bring the roast onto the counter and let it come up to room temperature, which can take up to two hours.

Pat the roast dry with paper towels, and brown all sides over medium-high heat in a large dutch oven. Reduce temperature if it starts to smoke. Remove roast to a plate.

\ingredients{
    1 & large onion \\
    1 & carrot \\
    1 & celery rib \\
    2 & whole garlic cloves \\
    1 & tablespoon kosher salt \\
}

Roughly chop vegetables, and add onions and salt to the pot, and cook until onions are just starting to brown. Add carrots, celery and garlic and cook for 5 more minutes.

Move the roast back to the pot, and add liquid until the level comes halfway up the roast. Better to have to little liquid than too much. Put in a 300\degree over for 5 hours, flipping the roast half way through.

Baked red bliss potatoes are a good accompaniment. Then can either be placed on the bare oven rack, or added to the pot in the last two hours.

\subsection{Classic}

\tip{
    You can blend the sauce and root vegetables for a thicker sauce.
}

\ingredients{
    1 & cup beef broth \\
    1 & cup red wine \\
    1 & spring fresh thyme \\
}

\subsection{Whiskey}

\tip{
    This one is my favorite.
}

\ingredients{
    1\sfrac{1}{2} & cup beef broth \\
     \sfrac{1}{2} & cup whiskey \\
}

\subsection{Yellow Curry}

\tip{
    Serve with peanuts and a Thai chili sauce over rice.
}

\ingredients{
    1 & cup chicken broth \\
    1 & cup white wine \\
    4 & tablespoons yellow \\
      & curry powder \\
}

\end{recipe}
