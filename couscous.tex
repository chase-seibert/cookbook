
\fullpageimage{couscous.jpg}{http://www.flickr.com/photos/stone-soup}

\section{Couscous Salad}
\begin{recipe}

\pre{
    Couscous has a tendency to clump together. Ideally, we want every kernel to be separate. Also, you want to serve the final salad as cold as possible. That means no lukewarm couscous.
}

\ingredients{
    1 & handful kosher salt \\
    3 & trays ice cubes \\
}

In one large metal bowl, combine ice, kosher salt, and enough water to cover the ice fully.

\ingredients{
    2 & boxes plain couscous \\
    1 & tablespoon butter \\
}

Follow the directions on the box to hydrate couscous. Immediately after the couscous is cooked, turn out into another metal bowl.

\columnbreak

\ingredients{
    \sfrac{1}{2} & cup Italian dressing \\
}

Mix the dressing into the couscous to help it separate. Fluff with a spatula, breaking up as many clumps as possible.

Place the metal bowl with the couscous into the bowl with the ice. Keep stirring every three minutes to break down the clumps.

\ingredients{
                6 & ounces feta cheese \\
                1 & orange bell pepper \\
                1 & yellow bell pepper \\
     \sfrac{1}{2} & large red onion \\
}

Dice the peppers and onion fine, and combine with feta cheese in the couscous bowl.

\ingredientsLeft{
    & Italian dressing \\
    & kosher salt \\
    & pepper \\
    & dried dill \\
}

Season to taste.

\columnbreak

\tip{
    The two large metal bowls with ice water is great for cooling just about anything quickly. It's kind of like a reverse double-boiler.
}

\tip{
    The boxed Couscous sold in the grocery store is actually pre-cooked, which is why it only takes a few minutes to hydrate. You can buy uncooked Couscous, which has larger grains and cooks like pasta.
}

\tip{
    When buying feta, I prefer the cryo-packed version with a little bit of juice sealed in. The free-standing feta in juice and the Saran-wrapped versions are dryer.
}

\end{recipe}
