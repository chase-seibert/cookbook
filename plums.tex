\section{Plums}
\begin{recipe}

\subsection{Plum Jam}

\pre{
    Every year we get about 20 pounds of plums off the tree in the front yard.
    This is what we do with the ones that we don't eat right away. I usually
    pick them around June 1; you want some of them to still be tart/underripe.
}

\ingredients{
      1 & packet Sure Jell (pectin) \\
      8 & cups plums, quartered and squished \\
      8 & cups sugar \\
}

Mix pectin, \sfrac{1}{2} cup of water and plums together, thoroughly dissolving
the pectin. Bring to high simmer over high heat in an oversized pot. Add
sugar all at once and bring to a boil that does not dissipate when you stir it.
You want to the mixture to reach 220\degree{}.

Remove from heat. Spoon into canning jars. Process for 10 minutes (see any canning recipe).

\subsection{Plum Preserves}

\pre{
    Makes a great topping for ice cream or cheesecake.
}

\ingredients{
      & halved plums, pit removed \\
      & simple sugar \\
}

Simple sugar is just equal parts sugar and water brought to a simmer. Pour over
plums in jars and process for 10 minutes.

\subsection{Plum Mixer}

\pre{
    This is great for drinks, either with carbonated water, simple sugar and
    lime, or as a cocktail mixer.
}

\ingredients{
      1 & pound whole plums \\
      1 & cup water \\
}

Place plums in a dutch oven with water. Cover and bring to a boil over
medium heat until water is reduced by half.  Strain into small canning jars. Process for
10 minutes.

\end{recipe}
