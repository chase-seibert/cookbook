\section{Neopolitan Pizza}
\begin{recipe}

\pre{
  You need to start the dough the night before, and perform two steps over at least a four hour period before dinner. 
  Pre-sliced pepperoni is better than whole. 
}

\ingredients{
    3\sfrac{1}{2} & ounces bread flour \\
    1 & teaspoon sugar \\
    1 & packet active dry yeast \\
    3\sfrac{1}{2} & ounces water \\
}

Combine. Seal and let it sit for 1 hour. Refrigerator over night, no more than 24 hours. 

\ingredients{
    8\sfrac{1}{2} & ounces bread flour \\
    5 & ounces water \\
    1 & tablespoon kosher salt \\
    1 & tablespoon olive oil  \\
}

Combine dough starter with other ingredients in the bowl of a stand mixer. Install the dough hook attachment. Mix for a full 15 minutes on medium speed. Remove dough hook and let rest in the bowl for 15 minutes. 

Roll into a ball. Put oil on your hands and pat the surface. Cover with plastic wrap and let rest for 90 minutes. 

Divide the dough into two pieces. Form into balls. The balls need to be pinched closed on the bottom. They need to be balloons. Covered with oil and plastic wrap. Let rest for 90 minutes. 30 minutes in to the rest, preheat the oven to 550\degree{} with a baking stone or steel. 

\ingredients{
    1 & small can whole San Marzano tomatoes \\
    & olive oil \\
    & kosher salt \\
    & fresh basil \\ 
}

In a bowl, crush tomatoes with your hand, add salt and olive oil and a little chopped basil. Don't cook the sauce.

To shape the final pizza, first put a dough into a bowl of extra flour then move back to the countertop. 
Use your fingertips to press lightly from the center to just before the edges. You don't want to compress the outer crust.
Stretch the dough by making the same two-hands spreading apart motion you would use to flatten a piece of paper. 

\ingredients{
    1 & pound block whole mozzarella \\
      & sliced pepperoni \\
}

Cut cheese into strips. Use \sfrac{1}{3} pound per pizza. 

Cook at 550\degree{} for 6 minutes or until there is black leopard spots on the top crust. You can keep a first pizza warm by lightly tenting with aluminum foil. Cool for at least 6 minutes.  

Let oven come back up to temp between pizzas. You can reset the oven thermometer by turning the oven off and on again.

\end{recipe}
