
\fullpageimage{pancakes.jpg}{http://www.flickr.com/photos/roboppy}

\section{Pancakes}
\begin{recipe}

\pre{
    I like to use buttermilk powder instead of liquid buttermilk, which always ends up spoiled in our fridge.
}

\ingredients{
               1 & cup flour (4.5 ounces) \\
               2 & teaspoons sugar \\
               1 & kosher teaspoon salt \\
    \sfrac{1}{2} & teaspoon baking powder \\
    \sfrac{1}{4} & teaspoon baking soda \\
}

Whisk in a large bowl to combine.

\ingredients{
    \sfrac{3}{4} & cup buttermilk \\
    \sfrac{1}{4} & cup whole milk \\
}

Combine in a small bowl or measuring cup.

\ingredients{
    2 & tablespoons butter \\
}

Melt in the microwave, and add to the milk mixture.

\columnbreak

\ingredients{
    1 & egg \\
}

Separate the white from the yolk. Add the yolk to the milk mixture, and stir to combine.

Preheat an electric griddle to 375\degree. You can also use a frying pan over medium heat.

Place the white in a stand mixer with a whisk attachment, and beat on high for three minutes, or until you get stiff peaks.

Pour the milk mixture into the dry mixture. With a spatula, stir just a few times, leaving some dry pockets.

Add the beater egg whites, and fold just a few times with the spatula. Better to stop before it's fully combined than to over-stir.

The batter should be thin enough to spread on its own. Add a little milk if you need to thin it out.

Add a generous amount of vegetable oil to the frying pan. You want a pool covering the entire surface, about half as high as a pancake.

Pour \sfrac{1}{4} cup of the batter onto an empty spot on the pan, and repeat until the pan is full.

After two or three minutes, the tops of the pancakes will start to show bubbles. Flip them, and cook for another three minutes.

Transfer the pancakes to a wire rack on a sheet pan in the over while the others finish.

\end{recipe}
